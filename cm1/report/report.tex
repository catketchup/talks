% Created 2018-12-27 Thu 19:58
% Intended LaTeX compiler: pdflatex
\documentclass[12pt, notitlepage, onecolumn, amsmath, amssymb, aps]{revtex4-1}

\usepackage[utf8]{inputenc}
\usepackage[T1]{fontenc}
\usepackage{graphicx}
\usepackage{grffile}
\usepackage{wrapfig}
\usepackage{rotating}
\usepackage[normalem]{ulem}
\usepackage{amsmath}
\usepackage{textcomp}
\usepackage{amssymb}
\usepackage{capt-of}
\usepackage{color}
\usepackage{hyperref}
\usepackage{dcolumn}
\usepackage{bm}
\usepackage{natbib}
\usepackage{float}
\usepackage{subcaption}

% \bibliographystyle{abbrvnat}
\bibliographystyle{unsrtnat}
%\date{\today}
\title{}
\hypersetup{
 pdfauthor={Yilun Guan},
 pdftitle={},
 pdfkeywords={},
 pdfsubject={},
 pdfcreator={Emacs 25.2.2 (Org mode 9.1.11)}, 
 pdflang={English}}
\begin{document}

%================== Begin of Header ========================
%\preprint{APS/123-QED}
\title{Research Summary}
\author{Hongbo Cai \\{\small Supervisor: Arthur Kosowsky}}
\affiliation{Department of Physics and Astronomy, University of Pittsburgh}
%\date{\today}
%\pacs{Valid PACS appear here}
%\keywords{Suggested keywords}
\maketitle
%================== End of Header =========================
\newcommand{\edit}[1]{\textcolor{red}{(#1)}}
\vspace{-1.5cm}
\section{Lensing reconstruction on the curved sky}
\label{sec:org8852578}


Cosmic Microwave Background(CMB) photons are gravitationlly lensed by the large scale mass distribution. We can use a first order perturbation to discribe this effect which is called weak gravitationl lensing of CMB.
CMB temperature map(T) and polarization map(E,B) are distorted by 
(what is gravitationl lensing)
Weak gravitationl lensing for the microwave background anisotropies
(dark matter)
in the projected gravtional potential
(premodial gravitationl waves)




There are several aspects where we see weak gravitationl lensing effects on CMB.
1.Weak gravitationl lensing modulates CMB temperature and polarization power spectra(2 point power spectrum). Statistical anisotropy is induced.
2.Lensing of CMB fields produces higher-order correlations between the multipole moments. Off-diagonal mode-coupling between map harmonics can be seen. The off-diagonal mode coupling is proportional to lensing potential.\cite{Hu:2001kj}
3.It generates B-mode polarization signal which confuses the signal from primordial gravitationl waves.\cite{Lewis:2006fu}

The lensing signal can help seperate premordial gravitationl waves(delensing) which can generate B-mode before the last scattering surface. It can also be used to help constrain cosmological parameters and lensing mass distribution. Thus, it is important to study weak gravitationl lensing of CMB.

To obtain projected gravitationl potential(lensing potential) and therefor the projected mass, we can take quadratic combinations of CMB fields. This process is called lensing reconstruction which is a story of measuring lensing. 

I will outline 

Since lensing is most sensitive to the projected potential at \(L<10^2\) or several degrees on the sky, we need consider the curvature of the sky. 


? how curved sky better than flat sky; compare between them? 
? isotropic? unlensed CMB multipoles are assumed to be Gaussian and statistically isotropic(explain)
? name of l and alm

In the section, I will present how to treat consisely the effect of gravitationl lensing on CMB temperature and polarization maps by constructing the full sky quadratic estimators of lensing potential and their noise. And I will introduce the simulation of lensed I will give preliminary results of my calculation 

Deriving the full sky minimum variance quadratic estimators of lensing potential from CMB temperature and polarization fields.

My work of this part is mainly about:
I am a member of both ACT and Simons

<<<<<<< HEAD
\section{Bias to CMB Lensing Reconstruction from Temperature Anisotropies due to Reionization kSZ}
\label{sec:org093d799}

In this work, I am investigating a bias to CMB lensing reconstruction from temperature anisotropies due to the reionization kSZ effect. kSZ effect is short for kinematic Sunyaev-Zel'dovich effect, which is the Doppler shift of CMB photons induced by Compton-scattering off moving electrons.\cite{Ferraro:2017fac}
The kSZ signal has its own intrinsic non-Gaussianity and its correlation with CMB lensing field is non-zero.\cite{Smith:2016lnt}. The kSZ signal can bring bias to CMB lensing field. Plus, the kSZ effect preserves the blackbody spectrum of CMB which means it cannot be removed by multifrequency seperation techniques.\cite{Smith:2009pn}


kSZ anisotropies can be produced when there are large fluctuation in electron density. There are two epochs when they can be produced: 1.a ``late-time'' contribution from redshifts\(z<~3\) when inhomogeneities are large due to gravitationl growth of structure 2. earlier during the epoch of reionization when hydrogen gets ionized again by the ultraviolet radiation of the first structures and th fluctuation electron density are caused by the fluctuations in the ionization fraction.\cite{Ferraro:2017fac} \cite{Alvarez:2015xzu}. It is expected to be correlated with lensing field.\cite{Ferraro:2017fac}

... and ... has investigated ...
They argued that ...
=======
\section{Bias to CMB Lensing Reconstruction from Temperature Anisotropies due to Reionization kSZ Effect}
\label{sec:org093d799}

In this work, I am investigating a bias to CMB lensing reconstruction from temperature anisotropies due to reionization kSZ effect.
kSZ effect is short for kinematic Sunyaev-Zel'dovich effect which is a Doppler shift of CMB photons induced by Compton-scattering off moving electrons.
>>>>>>> b3708dde2e6e4319548f1ebb5ce8f10d4e4a5c80



1.understand terms and calculations\cite{Ferraro:2017fac} \cite{Alvarez:2015xzu}
2.why the ksz signal due to fluctuations in the ionization fraction during reionization can cause a detectable squeezed limit trispectrum. (kSZ signal from reionization)\cite{Smith:2016lnt}

<<<<<<< HEAD
 Next generation ground-based CMB 
experiments 
=======
? ksz map is also can be considered as a random statistically isotropic?
In this work, I am investigating a bias to CMB lensing reconstruction from temperature anisotropies due to reionization kinematic Sunyaev-Zel'dovich (kSZ) effect. kSZ effect is a Doppler shift of CMB photons induced by Compton-scattering off moving electrons. Next generation ground-based CMB experiments 
>>>>>>> b3708dde2e6e4319548f1ebb5ce8f10d4e4a5c80

? compton scattering and Thomson scattering: Thomson scattering is the elastic scattering of electromagnetic radiation by a free charged particle, as described by classical electromagnetism. It is the low-energy limit of Compton scattering: the particle's kinetic energy and photon frequency do not change as a result of the scattering. This limit is valid as long as the photon energy is much smaller than the mass energy of the particle: , or equivalently, if the wavelength of the light is much greater than the Compton wavelength of the particle.

Reionization: At a redshift of about 10, hydrogen gets ionized again by the ultraviolet radiation of the first structures
kSZ: 



tSZ effect: CMB photons interact with electrons that have high energies due to their temperature
kSZ effect 
kSZ anisotropies: are produced in cosmological epochs during which there are large fluctuations in the electron density. ? (physical image)
``kSZ'' is used to refer to any blackbody temperature fluctuation arising from bulk motion integrated along the line of sight, including the Doppler effect.

? We refer to the Doppler effect as responsible for large-angle kSZ anisotropies.
? bulk velocity
? patchy reionization
? (B+C+E)


To lowest order (in both optical depth and velocity), the kSZ effect produced only temperature anisotropies, not polarization anisotropies. 



late-time kSZ: are present in galaxies and clusters due to the non-linear growth of structure
reionization kSZ: fluctuations in the electron density field are due to fluctuations in the ionization fraction, are also expected to be correlated with the matter density field and hence with CMB lensing. 

? non-linear growth of structure
? seperation of ``late-time'' ksz and ``reionization ksz''
? addition of ``reionization ksz'' map and ``temperature'' map
? how do I tell the difference between ``late-time'' kSZ and ``reionization'' kSZ?

? model of reionization kSZ
? CMB unlensed and lensed simulations
? websky reionization kSZ simulation and late-time kSZ(Dr.Trac might ask about this)

CMB-S3 and CMB-S4 lensing 
for a Stage 4 CMB experiment
Their results have neglected the kSZ signal from reionization. 
kSZ sigunal due to fluctuations in the ionization fraction during reionization can bring a detectable bias(squeezed limit trispectrum)
I am working on simulating the bias to lensing reconstruction from reionization kSZ effect. 


As a member of both the Atacama Cosmology Telescope (ACT) and the
Simons Observatory (SO), I am actively involved in the research and
analysis projects in the collaborations. A short summary of each of my
involvements is listed below.

<<<<<<< HEAD
Future
=======
One the other hand, I am simulating the bias; On the other hand, I am also trying to understand the bias quantitively. Also studying the reionization  ksz simulations give by ...;

Future:
So my work is mainly around lensing reconstruction.

1.Studying the bias to polarization reconstruction.
2.Primordial non-Gaussianity, collaborate with Yilun.
>>>>>>> b3708dde2e6e4319548f1ebb5ce8f10d4e4a5c80

\bibliography{references}

\end{document}