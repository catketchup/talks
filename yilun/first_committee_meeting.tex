% Created 2018-12-27 Thu 19:58
% Intended LaTeX compiler: pdflatex
\documentclass[12pt, notitlepage, onecolumn, amsmath, amssymb, aps]{revtex4-1}

\usepackage[utf8]{inputenc}
\usepackage[T1]{fontenc}
\usepackage{graphicx}
\usepackage{grffile}
\usepackage{wrapfig}
\usepackage{rotating}
\usepackage[normalem]{ulem}
\usepackage{amsmath}
\usepackage{textcomp}
\usepackage{amssymb}
\usepackage{capt-of}
\usepackage{color}
\usepackage{hyperref}
\usepackage{dcolumn}
\usepackage{bm}
\usepackage{natbib}
\usepackage{float}
\usepackage{subcaption}

% \bibliographystyle{abbrvnat}
\bibliographystyle{unsrtnat}
%\date{\today}
\title{}
\hypersetup{
 pdfauthor={Yilun Guan},
 pdftitle={},
 pdfkeywords={},
 pdfsubject={},
 pdfcreator={Emacs 25.2.2 (Org mode 9.1.11)}, 
 pdflang={English}}
\begin{document}

%================== Begin of Header ========================
%\preprint{APS/123-QED}
\title{Research Summary}
\author{Yilun Guan \\{\small Supervisor: Arthur Kosowsky}}
\affiliation{Department of Physics and Astronomy, University of Pittsburgh}
%\date{\today}
%\pacs{Valid PACS appear here}
%\keywords{Suggested keywords}
\maketitle
%================== End of Header =========================
\newcommand{\edit}[1]{\textcolor{red}{(#1)}}
\vspace{-1.5cm}
\section{Impacts of Primordial Magnetic Field on CMB Anisotropies}
\label{sec:org8852578}
One of the primary goals of the next generation CMB experiments is to
detect the primordial B-mode signal from the tensor perturbations
generated by inflation. A detection of such signal will be a solid
evidence of inflation and work as a ``smoking gun'' for various
inflationary models. The current best constraint on the
tensor-to-scalar ratio is \(r<0.056\) at \(95\%\) C.L. through a
combined analysis of Planck and BICEP2 \cite{Akrami:2018odb}, and this
bound is expected to be lowered to \(r \sim 10^{-3}\) by the upcoming
CMB experiments such as Simons Observertory \cite{Ade:2018sbj} and
CMB-S4 \cite{abazajian16}. However, the tensor perturbation from
inflation is not the only source of B-mode polarizations in
CMB. Effects such as topological defects \cite{lizarraga14}, cosmic
birefringence \cite{lee15}, and primordial magnetic field
\cite{shaw10} may also contribute to the B-mode polarizations in CMB
and can be mistaken as an inflationary tensor signal if not accounted
for properly.

In particular, magnetic fields with strengths of a few $\mu G$
coherent across galactic and cluster scales are ubiquitous in the
universe. There are evidences that they also exist in the
inter-cluster-space coherent on the Mpc scale
\cite{2010Sci...328...73N}. On the other hand, the physical origin of
the cosmic magnetic field is still poorly understood. One proposed
mechanism is that cosmic magnetic fields are produced in the very
early universe such as during inflation \cite{PhysRevD.37.2743} or in
phase transitions \cite{Vachaspati:1991nm} and act as the initial
seeds to the observed magnetic fields in galaxies and galaxy
clusters. If existed, primordial magnetic fields (PMF) would impact
both cosmology and particle physics leaving imprints on different
observables such as the CMB anisotropies and the matter power
spectrum. On the other hand, PMF is poorly constrained by the existing
observations. The 2015 Planck bound on the scale invariant PMF is
\(B_{1 \mathrm{Mpc}} < 2 \mathrm{nG}\)
\cite{collaboration15}. However, it has been suggested that a field of
\(B \sim 1\mathrm{nG}\) can generate detectable patterns in CMB
anisotropies \cite{shaw10}. Therefore, it's important to evaluate the
impacts of PMF on the science goals of the upcoming CMB
experiments. In particular, how the B-mode measurements will be
affected by the PMF.

PMF affects CMB power spectra primarily through metric perturbation
sourced by its stress-energy tensor and through the Lorentz force felt
by the baryons in the plasma \cite{Mack:2001gc}. Prior to the neutrino
decoupling, the universe is dominated by a tightly coupled radiative
fluid that hinders the development of any significant anisotropic
stress. Therefore, the total anisotropic stress is dominated by that
from the PMF. This leads to the so called ``passive'' mode of
PMF. After the neutrino decoupling, the anisotropic stress from the
neutrinos will compensate that of the PMF, and this leads to the
``compensated'' mode of PMF \cite{shaw10}.

PMF sources all modes of metric perturbation: scalar, vector and
tensor mode. In particular, the passive tensor mode contributes
significantly to the B-mode polarization for low-$\ell$ similar to
what one would expect from an inflationary B-mode signal. It has been
shown that the tensor mode signal from a 1.08 nG PMF looks identical
to an $r=0.0042$ inflationary B-mode signal \cite{renzi18} in CMB
B-mode power spectrum. On the other hand, PMF also generates a vector
mode that causes dominant B-mode polarization anisotropies that
survives well below the Silk scale. Knowing the extent of which the PMF
signal can confuse us as an inflationary signal without having its
other imprints such as the vector mode signal being detected is very
important for the upcoming CMB experiments, but the answer is unclear.

In this study, we simulate the observed CMB angular power spectra for
different sets of hypothetical experimental settings with varying
noise levels. The simulated power spectra are based on a fiducial
model with the best-fit cosmology from the latest Planck result
\cite{Aghanim:2018eyx} adding in a non-zero
tensor-to-scalar ratio $r$. We fit the simulated observations to two
different cosmological models: one with a non-zero $r$ but zero PMF
(termed $\Lambda$CDM+r hereafter), and one with a non-zero PMF but
with $r=0$ (termed $\Lambda$CDM+PMF hereafter). We find the best-fit
cosmology using an MCMC-based maximum likehood approach with the
likelihood calculated using the exact likelihood method following
Ref. \cite{Perotto:2006rj}. We calculate the maximum likelihood value
that corresponds to the best-fit cosmology for both $\Lambda$CDM+r
model and $\Lambda$CDM+PMF model. A significantly lower likelihood for
the $\Lambda$CDM+PMF model indicates a good distinguishability between
the two cosmological models, but a close match is the sign of a
potential confusion between the two cosmological models. We repeat
this comparison for $r=0.01$, $r=0.004$, $r=0.003$, and $r=0.001$ to
understand the extent of the confusion for different science targets.

Preliminary results show that the degeneracy between the
$\Lambda$CDM+r model and the $\Lambda$CDM+PMF model is broken in the
noiseless limit, but for a SO-like noise level ($\sim 4\mu$K arcmin)
the two models are highly degenerate in the power spectra. Therefore,
to be convinced of a detection of the primordial B-mode signals, SO
will need to perform additional checks beyond the power spectra level
such as the non-gaussianity check \cite{Lewis:2004ef} and the
Faraday's rotation check \cite{Kosowsky:1996yc} to rule out the
possibility of seeing the PMF instead.

% Fig. \ref{fig:b-mode} shows a comparison of the B-mode polarizations
% generated by the PMF, primordial gravitational wave and lensing, and
% the degeneracy of PMF and primordial gravitationl wave is evident at
% low-$\ell$. The forecast sensitivities of some current and upcoming
% observatory are also shown in Fig. \ref{fig:b-mode} together with
% the theory spectrum generated with PMF. Clearly, if we were to
% discover a B-mode polarization signal in CMB-S4 experiments, it's
% important to check whether it may be a signal from the PMF before
% claiming a discovery of primordial gravitational wave.

% \begin{figure*}[p]
%   \centering
%   \includegraphics[width=\textwidth]{plots/power_spectra}
%   \caption{This plot shows how passive modes and compensated modes
%     contribute to $C_\ell^{\rm TT}$ (upper left), $C_\ell^{\rm EE}$
%     (upper right), $C_\ell^{\rm TE}$ (lower left) and
%     $C_\ell^{\rm BB}$ (lower right). This plot is generated by the
%     public available MagCAMB codes \cite{zucca17}.}
%   \label{fig:power}
% \end{figure*}

% \begin{figure*}[p]
%   \centering
%   \includegraphics[width=\textwidth]{plots/b-mode}
%   \caption{left: A comparison of the contributions to B-mode
%     polarization in CMB from PMF, primordial gravitational wave and
%     lensing}
%   \label{fig:b-mode}
% \end{figure*}

\section{Involvements in the collaboration}
\label{sec:org093d799}
As a member of both the Atacama Cosmology Telescope (ACT) and the
Simons Observatory (SO), I am actively involved in the research and
analysis projects in the collaborations. A short summary of each of my
involvements is listed below.

\begin{itemize}
\item \textbf{Constraining Cosmic Birefringence with ACT Data:} Cosmic
  birefringence refers an effect where the propagation of different
  polarization states is different. This leads to a rotation of CMB
  polarization from E-modes to B-modes. It can be caused by a
  parity-violating physics in the early universe such as a coupling
  between photons and an axion-like pseudoscalar field via the
  Chern-Simons interaction term \cite{Harari:1992ea,
    Carroll:1998zi}. It can also be caused by primordial magnetic
  fields via Faraday rotation \cite{Kosowsky:1996yc}. The
  inhomogeneities in the pseudoscalar fields or the PMF produce
  anisotropies in the rotation angle that can be reconstructed from
  the CMB trispectrum similar to weak lensing
  \cite{2009PhRvD..79l3009Y}. Using the ACT data from s14 to s16, we
  perform a full-sky reconstruction of the rotational field with
  quadratic estimators to constrain the cosmic birefringence.
  Preliminary study shows that we expect to improve the current best
  constraint by an order of magnitude.
\item \textbf{Search for Fast Radio Bursts in ACT Data:} Fast radio
  bursts (FRB) are the mysterious bursts in radio emission that last
  for only a few milliseconds \cite{lorimer07}. They have been
  observed in various radio telescopes \cite{petroff16}, but their
  physical origin remains a mystery. Preliminary estimates show that
  we may potentially observe them in the ACT telescope as a short
  glitch that affects no more than two adjacent feedforns, and they
  are expected to occur on the order of a few times each month. We
  perform a matched-filter search for potential FRB signals in ACT
  data collected in 2016. Success or failure to find such signals in
  microwave band will be equally important in helping us constrain
  the underlying models of FRBs.
\item \textbf{Automate Data Cuts with Machine Learning:} In order to
  obtain high quality CMB maps, an expert knowledge based data cuts
  pipeline is implemented in ACT that selects good chunks of data and
  reliable list of detectors for mapping \cite{duenner12}. However, it
  requires too much human intervention due to the large number of
  parameters that one needs to tune for each array and each season.
  This makes it difficult to scale to the newer seasons of ACT and the
  upcoming SO. We explore the use of machine learning methods to
  automate the data cuts process. We train our models using the
  existing data cuts that are generated by expert
  knoweldge. Preliminary results show that one can achieve $>90\%$
  accuracy with machine learning without any human intervention.
\end{itemize}
My other responsibilities in the collaborations involve generating
data cuts and atmosphere calibrations for ACT and developing a similar
pipeline for SO.

%\section{Search of Fast Radio Burst in Microwave Experiments}
%\section{Data Characterization and Time Domain Astronomy with ACTPol}

% I am also involved in a data-oriented project in the ACT Collaboration
% working on the data cuts and flagging pipeline. The primary sources
% of contamination to ground-based microwave experiments include
% atmosphere effects, thermal drift, point sources, electronic glitches,
% etc. To minimize their impacts on CMB mapping, an elaborate data
% flagging pipeline based on multi-frequency analysis is designed
% \cite{duenner12}.

% The time-order data (TOD) is first splitted into groups in the fourior
% space. The low frequency data is mainly affected by the atomosphere
% noise and thermal drift. A common mode analysis is then performed, and
% bad detectors are flagged if they deviate from the common mode of the
% detector array. The high frequency part of data is mainly affected by
% the short-duration glitches such as electronics glitches and cosmic
% rays, and they are identified with a high pass filter and a gaussian
% template \cite{duenner12}.

% One of the main drawbacks of this existing pipeline is that it
% requires too much human intervention due to the large number of
% parameters that one needs to tune for each array and each season in
% order to produce high quality CMB maps. A large part of my efforts in
% this area involves exploring smarter flagging algorithms that reduce
% the amount of human intervention in the pipeline. In particular, I am
% exploring various machine learning methods that can be applied that we
% can train on the previous flagging results and make automatic flagging
% for new data without manual parameter tuning. This work is also timely
% because the Simons Observatory will collect an order of magnitude more
% data than ACT, so a smarter data flagging algorithm is due.

% Another part of my efforts in this area involves characterizing the
% physical causes of the various glitches that are identified by the
% data flagging pipeline. For example, a glitch that results from a
% point source tends to leave a track on the detector array that
% corresponds to the scanning pattern. A glitch that results from cosmic
% ray hits tends to be localized to a few neighboring feedhorns with an
% impact that lasts \(10 \sim 100\) ms depending on whether it hits the
% detector or the substrate. Fig. \ref{fig:glitch} illustrates the
% different types of glitches and shows an example of glitches resulted
% from a point source crossing the array.

% \begin{figure*}[hp]
% \centering
% \begin{subfigure}{.5\textwidth}
%   \centering
%   \includegraphics[width=0.8\linewidth]{plots/data/51/f70a05-cab2-428a-96a3-912403f31a5d/overview.png}
%   \caption{}
% \end{subfigure}%
% \begin{subfigure}{.5\textwidth}
%   \centering
%   \includegraphics[width=0.8\linewidth]{plots/data/8b/fe6c89-7252-4512-b8d2-24d130ec1cdc/point_source.png}
%   \caption{}  
% \end{subfigure}
% \caption{(a) An overview of different types of glitches that can occur
%   on the array; (b) A plot of the tracks of the glitches from the data
%   when the telescope array scans across a planet. It leaves a
%   horizontal pattern consistent with the scanning pattern.}
% \label{fig:glitch}
% \end{figure*}

% A good characterization of the glitches not only provides us insights
% into the pathological issues of the array but also enables us to dive
% into the time-domain astronomy such as looking for transient
% sources. Fast radio bursts (FRB) are the mysterious bursts in radio
% emission that lasts for only a few milliseconds \cite{lorimer07}, and
% they have been observed in various radio telescopes \cite{petroff16}.
% A rough estimate (see Fig. \ref{fig:frb} for example) shows that one
% may be able to observe them in the ACT telescope as an extremely short
% glitch that affects no more than two adjacent feedforns. With a good
% understanding of the short glitches identified in the ACT array, we
% are trying to find if any of the glitches correspond to an FRB
% signal. An observation of FRB in microwave is of particular importance
% as it allows us to constrain the dispersion relation of the FRBs.
% \begin{figure*}[p]
% \centering
% \begin{subfigure}{.5\textwidth}
%   \centering
%   \includegraphics[width=0.7\linewidth]{plots/data/0c/e29feb-1cbf-4bcb-bb0c-e72bf762c71e/FRBSignal_bw_2pix.pdf}
%   \caption{}  
% \end{subfigure}%
% \begin{subfigure}{.5\textwidth}
%   \centering
%   \includegraphics[width=\linewidth]{plots/data/fd/eb307f-c33c-40d4-95cc-40fa939d3740/SNR_bw_2pix.pdf}
%   \caption{}  
% \end{subfigure}
% \caption{(a) A simulated beam from an FRB event falling onto the
%   detector array. It shows that it's most likely to affect less than
%   three detectors on the array. (b) An estimation of the
%   signal-to-noise to ratio for FRB events of various brightness. The
%   dotted line indicates the noise level of the detectors. This
%   indicates that one may be able to detect the FRB signals in two
%   adjacent detectors. }
% \label{fig:frb}
% \end{figure*}

\bibliography{../../../bib/references}
\end{document}